\documentclass{article}
%-------------------------------------------------

\usepackage{amsmath}
\usepackage{amssymb}
\usepackage{amsfonts}

\usepackage{color}

\DeclareMathOperator{\sinc}{sinc}

%-------------------------------------------------
\begin{document}
%-------------------------------------------------

%------------------------
\section*{Michelson Interferometer}

Consider a plane wave propagating in the $\hat{n}$ direction
\begin{equation}
    h_{\mu\nu} = h_{\mu\nu}(t-n_i r^i)
\end{equation}
Furthermore, assume the arm is pointed in the $\hat{e}$ direction, so that the component of strain along the arm is given by
\begin{equation}
    h = h_{\mu\nu}e^\mu e^\nu = h(t-n_i r^i)
\end{equation}
We further consider motion only along the $\hat{e}$ direction, so we are interest in
\begin{equation}
    h = h(t-n_i e^i x) = h(t-n_e x)
\end{equation}
where $x$ is the distance along the arm and we've defined $n_e\equiv n_i e^i$.
Now, we know light will follow a null geodescic, so that
\begin{equation}
    dt^2 = (1+h)dx^2
\end{equation}
Furthermore, we assume there is a single fourier component of the GW strain, so that
\begin{equation}
    h = h_0 e^{i\omega(t-n_e x)}
\end{equation}

Consider the outbound trip.
In this case, $dx/dt > 0$ and
\begin{equation}
    dt = \sqrt{1+h}\, dx = (1 + \frac{1}{2}h + O(h^2))dx
\end{equation}
We define $d\tau = dt-dx$ and note that $\tau$ denotes the phase of light as it propagates down the arm.
Worldlines of constant $\tau$ represent worldlines of constant phase.
We assume $h\ll1$, and obtain
\begin{align}
    \int\limits_{t_0-x_0}^{T+t_0 - (L+x_0)} d\tau\, e^{-i\omega\tau} & = \frac{h_0}{2} \int\limits_{x_0}^{L+x_0} dx\, e^{i\omega(1-n_e)x} \\
    \frac{e^{-i\omega(t_0-x_0)}}{-i\omega}\left(e^{-i\omega(T-L)} - 1\right) & = \frac{h_0}{2}\frac{e^{i\omega(1-n_e)x_0}}{i\omega(1-n_e)}\left( e^{i\omega(1-n_e)L} - 1 \right)
\end{align}

For the return trip, we know $dx/dt < 0$ and define $d\tau^\prime = dt + dx$.
Again, $\tau^\prime$ is the phase of the light on the return path.
This yields
\begin{align}
    \int\limits_{T+t_0+L+x_0}^{T^\prime+t_0+x_0} d\tau^\prime\, e^{-i\omega\tau^\prime} & = -\frac{h_0}{2} \int\limits_{L+x_0}^{x_0} dx\, e^{-i\omega(1+n_e)x} \\
    \frac{e^{-i\omega(t_0+x_0)}}{-i\omega}\left(e^{-i\omega T^\prime} - e^{-i\omega(T+L)}\right) & = - \frac{h_0}{2} \frac{e^{-i\omega(1+n_e)x_0}}{-i\omega(1+n_e)}\left( 1 - e^{-i\omega(1+n_e)L}\right)
\end{align}
We are interested in the change in round-trip time relative to the unperturbed state ($(T^\prime - 2L)/T \sim h_0$), and therefore write
\begin{align}
    e^{-i\omega(T^\prime-2L)} - e^{-i\omega(T-L)} & = -\frac{h_0}{2} e^{i\omega(t_0+x_0+2L)} \frac{e^{-i\omega(1+n_e)x_0}}{1+n_e}\left(1-e^{-i\omega(1+n_e)L}\right) \\
    e^{-i\omega(T-L)} & = 1 - \frac{h_0}{2}e^{i\omega(t_0-x_0)}\frac{e^{i\omega(1-n_e)x_0}}{1-n_e}\left(e^{i\omega(1-n_e)L} - 1\right)
\end{align}
which yields
\begin{align}
    e^{-i\omega(T^\prime-2L)} - 1 & = \frac{h_0}{2}\left( e^{i\omega(t_0-x_0 + (1-n_e)x_0)}\left(\frac{1-e^{i\omega(1-n_e)L}}{1-n_e}\right) \right. \\
                                  & \quad \quad \quad \quad \left. - e^{i\omega(t_0+x_0+2L - (1+n_e)x_0} \left(\frac{1-e^{-i\omega(1+n_e)L}}{1+n_e}\right) \right) \\
                                  & = \frac{h_0}{2} e^{i\omega(t_0-n_ex_0)} \left( \frac{1-e^{i\omega(1-n_e)L}}{1-n_e} - e^{2i\omega L}\frac{1-e^{-i\omega(1+n_e)L}}{1+n_e} \right)
\end{align}
We now assume $T^\prime - 2L \ll 1/\omega$ and obtain
\begin{equation}
    \frac{T^\prime - 2L}{L} = \frac{h_0 e^{i\omega(t_0-n_ex_0)}}{-2i\omega L} \left(\frac{1-e^{i\omega(1-n_e)L}}{1-n_e} - e^{2i\omega L}\frac{1-e^{-i\omega(1+n_e)L}}{1+n_e} \right)
\end{equation}

We recognize $h_0 e^{i\omega(t_0-n_ex_0)}$ as the strain projected along the arm at the space-time coordinates corresponding to the start of the round-trip, and can therefore express this as
\begin{equation}
    \frac{\delta T}{L} = \frac{1}{-2i\omega L} \left(\frac{1-e^{i\omega(1-n_e)L}}{1-n_e} - e^{2i\omega L}\frac{1-e^{-i\omega(1+n_e)L}}{1+n_e} \right) e_i e_j h^{ij}
\end{equation}
This yields the transfer function from the astrophysical signal to the observed fractional change in round-trip travel time. 
Furthermore, if we define
\begin{equation}
    D(\omega, n_e) \equiv \frac{1}{-2i\omega L} \left(\frac{1-e^{i\omega(1-n_e)L}}{1-n_e} - e^{2i\omega L}\frac{1-e^{-i\omega(1+n_e)L}}{1+n_e} \right),
\end{equation}
we can complactly report the output of an interferometer as
\begin{align}
    \delta V & = \frac{1}{2}\left( D(\omega, n_\alpha e_x^\alpha) e_x^i e_x^j - D(\omega, n_\beta e_y^\beta) e_y^i e_y^j \right) h_{ij} \\
             & = D^{ij} h_{ij}
\end{align}
where we have inserted a factor of $1/2$ so that $\delta V\sim h$ instead of $\sim 2h$ (the maximum of the antenna patterns is 1).
This yields standard expressions for the antenna patterns as follows
\begin{align}
    F_+ & = D_{ij} e_+^{ij} \\
    F_\times & = D_{ij} e_\times^{ij}
\end{align}
with $e_+$ and $e_\times$ defining the polarization tensors for the $+$ and $\times$ polarizations, respectively.
This gives us a coordinate-independent expression for the antenna patterns which includes frequency-dependent effects for a Michelson interferometer.

%-----------
\subsection*{Matt Evan Memorial Expression for $D(\omega, n_e)$}

We note that with some elbow grease, we can massage the expression for $D(\omega, n_e)$ into

\begin{equation}
    D(\omega, n_e) = \frac{e^{i\omega L}}{1-n_e^2}\left( \sinc(\omega L) + \frac{n_e}{-i\omega L} \left( \cos(\omega L) - e^{-i\omega n_e L}\right)\right)
\end{equation}
where $\sinc x = x^{-1} \sin x$.

%------------------------
\section*{Fabry-Perot cavity}

Let us now consider the response of the reflected light from a Fabry-Perot cavity. 
In this case, we have 3 electric fields of interest: the incident field $E_\mathrm{inc}$, the reflected field $E_\mathrm{ref}$, and the circulating field just within the cavity $E$.
\textcolor{red}{Add a figure to clarify what we mean?}
We expect these to be related by
\begin{align}
    E & = i t_a E_\mathrm{inc} + r_a r_b e^{-\alpha -i\phi} E \\
    E_\mathrm{ref} & = r_a E_\mathrm{inc} + i t_a r_b e^{-\alpha -i\phi} E
\end{align}
where mirror $a$ is the one on which the incident light falls, $\alpha$ measures the losses within the cavity, and $\phi$ represents the phase accumulated by the light during one round-trip down the cavity and back.
This tells us that 
\begin{equation}
    \frac{E_\mathrm{ref}}{E_\mathrm{inc}} = r_a + i t_a r_b e^{-\alpha -i\phi} \frac{i t_a}{1 - r_a r_b e^{-\alpha -i\phi}}
\end{equation}
Now, in the transverse-traceless gage, coordinate time is proper time for all observers.
This means that the amount of phase accumulated by the light in one round trip is $\nu T^\prime$, where the light's angular frequency is $\nu$ and $T^\prime$ is defined by our analysis of the Michelson case.
\begin{align}
    e^{i\phi} & = e^{-2i\nu L}\left(1 + h_0 e^{i\psi_0} (-i\omega L)D(\omega, n_e)\right)^{\nu/\omega} \\
              & \approx e^{-2i\nu L - i\nu L h_0 e^{i\psi_0} D(\omega, n_e)}
\end{align}
where $\psi_0 = \omega(t_0 - n_e x_0)$. 
Inserting this into our expression for $E_\mathrm{ref}$ yields
\begin{equation}
    \frac{E_\mathrm{ref}}{E_\mathrm{inc}} = r_a - \frac{t_a^2 r_b e^{-\alpha - 2i\nu L}}{1 - r_a r_b e^{-\alpha-2i\nu L}} \left( 1 + \frac{i\nu L D(\omega, n_e) h_0 e^{i\psi_0}}{1 - r_a r_b e^{-\alpha-2i\nu L}} + O(h_0^2) \right)
\end{equation}

We expect the interferometer's readout to be
\begin{align}
    \delta V & = \frac{\left(E_\mathrm{ref}\right)_x - \left(E_\mathrm{ref}\right)_y}{E_\mathrm{inc}} \\
             & = \frac{r_a - t_a r_b e^{-\alpha -2i\nu L}}{1 - r_a r_b e^{-\alpha-2i\nu L}}\left(\frac{i\nu L}{1 - r_a r_b e^{-\alpha-2i\nu L}}\right)\left( D(\omega, n_\alpha e_x^\alpha) e_x^i e_x^j - D(\omega, n_\beta e_y^\beta) e_y^i e_y^j \right) h^{ij}
\end{align}
where we've assumed $L_x=L_y=L$ and $|\vec{x}_0 - \vec{y}_0|\omega/c \lesssim 10^{-4} \ll 1$ so that the starting phases of the gravitational wave at both the $x$ and $y$ arms are equal. 
In this case, we again see that the output can be written in the form
\begin{equation}
    \delta V = H(\nu L) D(\omega, \hat{n}, \hat{e}_x, \hat{e}_y)_{ij} h^{ij}
\end{equation}
where $D_{ij}$ is identical to the Michelson case. 
We assume the output is correctly calibrated so that $H(\nu L)$ is correctly compensated.
This means that we need only concern ourselves with $D_{ij}$ when projecting signals into the detector output.

Furthermore, we expect the only significant directional dependence will come from the long Fabry-Perot cavities and therefore assume that $D_{ij}$ will capture all the directional dependence regardless of what other cavities the signal is passed through before measurement (e.g.; signal-recyling).

%-------------------------------------------------
\end{document}
