\documentclass{article}
%-------------------------------------------------

\usepackage{amsmath}
\usepackage{amsfonts}

\usepackage{color}

\DeclareMathOperator{\sinc}{sinc}

%-------------------------------------------------
\begin{document}
%-------------------------------------------------

%------------------------
\section*{Michaelson Interferometer}

Consider a plane wave propagating in the $\hat{n}$ direction
\begin{equation}
    h_{\mu\nu} = h_{\mu\nu}(t-n_i r^i)
\end{equation}
Furthermore, assume the arm is pointed in the $\hat{e}$ direction, so that the component of strain along the arm is given by
\begin{equation}
    h = h_{\mu\nu}e^\mu e^\nu = h(t-n_i r^i)
\end{equation}
We further consider motion only along the $\hat{e}$ direction, so we are interest in
\begin{equation}
    h = h(t-n_i e^i x) = h(t-n_e x)
\end{equation}
where $x$ is the distance along the arm and we've defined $n_e\equiv n_i e^i$.
Now, we know light will follow a null geodescic, so that
\begin{equation}
    dt^2 = (1+h)dx^2
\end{equation}
Furthermore, we assume there is a single fourier component of the GW strain, so that
\begin{equation}
    h = h_0 e^{i\omega(t-n_e x)}
\end{equation}

Consider the outbound trip.
In this case, $dx/dt > 0$ and
\begin{equation}
    dt = \sqrt{1+h}dx = (1 + \frac{1}{2}h + O(h^2))dx
\end{equation}
We define $d\tau = dt-dx$ and note that $\tau$ denotes the phase of light as it propagates down the arm.
Worldlines of constant $\tau$ represent worldlines of constant phase.
We assume $h\ll1$, and obtain
\begin{align}
    \int\limits_{t_0-x_0}^{T+t_0 - (L+x_0)} d\tau\, e^{-i\omega\tau} & = \frac{h_0}{2} \int\limits_{x_0}^{L+x_0} dx\, e^{i\omega(1-n_e)x} \\
    \frac{e^{-i\omega(t_0-x_0)}}{-i\omega}\left(e^{-i\omega(T-L)} - 1\right) & = \frac{h_0}{2}\frac{e^{i\omega(1-n_e)x_0}}{i\omega(1-n_e)}\left( e^{i\omega(1-n_e)L} - 1 \right)
\end{align}

For the return trip, we know $dx/dt < 0$ and define $d\tau^\prime = dt + dx$.
Again, $\tau^\prime$ is the phase of the light on the return path.
This yields
\begin{align}
    \int\limits_{T+t_0+L+x_0}^{T^\prime+t_0+x_0} d\tau^\prime\, e^{-i\omega\tau^\prime} & = -\frac{h_0}{2} \int\limits_{L+x_0}^{x_0} dx\, e^{-i\omega(1+n_e)x} \\
    \frac{e^{-i\omega(t_0+x_0)}}{-i\omega}\left(e^{-i\omega T^\prime} - e^{-i\omega(T+L)}\right) & = - \frac{h_0}{2} \frac{e^{-i\omega(1+n_e)x_0}}{-i\omega(1+n_e)}\left( 1 - e^{-i\omega(1+n_e)L}\right)
\end{align}
We are interested in the change in round-trip time relative to the unperturbed state ($(T^\prime - 2L)/T \sim h_0$), and therefore write
\begin{align}
    e^{-i\omega(T^\prime-2L)} - e^{-i\omega(T-L)} & = -\frac{h_0}{2} e^{i\omega(t_0+x_0+2L)} \frac{e^{-i\omega(1+n_e)x_0}}{1+n_e}\left(1-e^{-i\omega(1+n_e)L}\right) \\
    e^{-i\omega(T-L)} & = 1 - \frac{h_0}{2}e^{i\omega(t_0-x_0)}\frac{e^{i\omega(1-n_e)x_0}}{1-n_e}\left(e^{i\omega(1-n_e)L} - 1\right)
\end{align}
which yields
\begin{align}
    e^{-i\omega(T^\prime-2L)} - 1 & = \frac{h_0}{2}\left( e^{i\omega(t_0-x_0 + (1-n_e)x_0)}\left(\frac{1-e^{i\omega(1-n_e)L}}{1-n_e}\right) \right. \\
                                  & \quad \quad \quad \quad \left. - e^{i\omega(t_0+x_0+2L - (1+n_e)x_0} \left(\frac{1-e^{-i\omega(1+n_e)L}}{1+n_e}\right) \right) \\
                                  & = \frac{h_0}{2} e^{i\omega(t_0-n_ex_0)} \left( \frac{1-e^{i\omega(1-n_e)L}}{1-n_e} - e^{2i\omega L}\frac{1-e^{-i\omega(1+n_e)L}}{1+n_e} \right)
\end{align}
We now assume $T^\prime - 2L \ll 1/\omega$ and obtain
\begin{equation}
    \frac{T^\prime - 2L}{L} = \frac{h_0 e^{i\omega(t_0-n_ex_0)}}{-2i\omega L} \left(\frac{1-e^{i\omega(1-n_e)L}}{1-n_e} - e^{2i\omega L}\frac{1-e^{-i\omega(1+n_e)L}}{1+n_e} \right)
\end{equation}

We recognize $h_0 e^{i\omega(t_0-n_ex_0)}$ as the strain projected along the arm at the space-time coordinates corresponding to the start of the round-trip, and can therefore express this as
\begin{equation}
    \frac{\delta T}{L} = \frac{1}{-2i\omega L} \left(\frac{1-e^{i\omega(1-n_e)L}}{1-n_e} - e^{2i\omega L}\frac{1-e^{-i\omega(1+n_e)L}}{1+n_e} \right) e_i e_j h_{ij}
\end{equation}
This yields the transfer function from the astrophysical signal to the observed fractional change in round-trip travel time. 
Furthermore, if we define
\begin{equation}
    D(\omega, n_e) \equiv \frac{1}{-2i\omega L} \left(\frac{1-e^{i\omega(1-n_e)L}}{1-n_e} - e^{2i\omega L}\frac{1-e^{-i\omega(1+n_e)L}}{1+n_e} \right),
\end{equation}
we can complactly report the output of an interferometer as
\begin{align}
    \delta V & = \frac{1}{2}\left( D(\omega, n_\alpha e_x^\alpha) e_x^i e_x^j - D(\omega, n_\beta e_y^\beta) e_y^i e_y^j \right) h_{ij} \\
             & = D^{ij} h_{ij}
\end{align}
where we have inserted a factor of $1/2$ so that $\delta V\sim h$ instead of $\sim 2h$ (the maximum of the antenna patterns is 1).
This yields standard expressions for the antenna patterns as follows
\begin{align}
    F_+ & = D_{ij} e_+^{ij} \\
    F_\times & = D_{ij} e_\times^{ij}
\end{align}
with $e_+$ and $e_\times$ defining the polarization tensors for the $+$ and $\times$ polarizations, respectively.
This gives us a coordinate-independent expression for the antenna patterns which includes frequency-dependent effects for a Michelson interferometer.

%-----------
\subsection*{Matt Evan's memorial expression for $D(\omega, n_e)$}

We note that with some elbow grease, we can massage the expression for $D(\omega, n_e)$ into

\begin{equation}
    D(\omega, n_e) = \frac{e^{i\omega L}}{1-n_e^2}\left( \sinc(\omega L) + \frac{n_e}{-i\omega L} \left( \cos(\omega L) - e^{-i\omega n_e L}\right)\right)
\end{equation}
where $\sinc x = x^{-1} \sin x$.

%------------------------
\section*{Fabry-Perot cavity}

\textcolor{red}{
Derive the change in the reflected light's phase coming out of a Fabry-Perot cavity and how it depends on the gravitaitonal-wave strain incident on the detector.
Should be able to derive an analogous expression for $D_{ij}$, which we should then use in all our calculations.
Note, the actual IFO readout is more complicated than this, but the majority of the strain's influence will be felt in the long Fabry-Perot cavities and therefore they should dominate $D_{ij}$.
}


%-------------------------------------------------
\end{document}
